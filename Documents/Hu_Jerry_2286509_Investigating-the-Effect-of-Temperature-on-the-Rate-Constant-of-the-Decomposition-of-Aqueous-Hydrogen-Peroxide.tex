% Imports
\documentclass[hyphens]{article}
\usepackage{graphicx, float}
\graphicspath{{Images/}}
\usepackage[letterpaper, top=2cm, bottom=2cm, left=2cm, right=2cm, heightrounded]{geometry}
\usepackage[numbers]{natbib}
\usepackage{url}
\usepackage{hyperref}
\hypersetup{breaklinks=true}
\usepackage{caption}
\usepackage[utf8]{inputenc}
\usepackage[version=4]{mhchem}
\usepackage{multirow}
\usepackage{multicol}
\usepackage{titling}
\usepackage{amsmath}
\usepackage{setspace}
\usepackage{xcolor}
\usepackage{array}
\usepackage{enumitem}
\usepackage{tablefootnote}
\newlist{titemize}{itemize}{1}
\setlist[titemize]{nosep, label=--}
\usepackage{tikz}
\usepackage{amssymb}
\usepackage{pdflscape}
\usepackage{soul}
\usepackage{soul}
\usetikzlibrary{calc}

% Doc settings
\title{\textbf{Investigating the Effect of Temperature on the Rate Constant of the Decomposition of Aqueous Hydrogen Peroxide.}}
\date{January 28, 2026}
\setlength{\parindent}{3em}

% Word count automation & reminder
\newcommand{\realwordcount}{
  \immediate\write18{wc=$(texcount "/Users/jerryhu/Library/CloudStorage/OneDrive-WhiteRockChristianAcademy/12/Chemistry HL/Chemistry IA/Documents/Hu_Jerry_2286509_Investigating-the-Effect-of-Temperature-on-the-Rate-Constant-of-the-Decomposition-of-Aqueous-Hydrogen-Peroxide.tex" | grep "Words in text" | sed -E "s/[^0-9]+//g"); if [ -f wordcount_flag.txt ]; then prev=$(cat wordcount_flag.txt); else prev=0; fi; if [ "$wc" -gt 3000 ] && [ "$prev" -eq 0 ]; then osascript -e 'display alert "Word Count Exceeded" message "Your IA is now over 3000 words."'; echo 1 > wordcount_flag.txt; elif [ "$wc" -le 3000 ]; then echo 0 > wordcount_flag.txt; fi; echo $wc > wordcount.tex}
  \input{wordcount.tex}
}

% Document
\begin{document}

\begin{titlepage}
  \centering
  \vspace*{2.75in}

  {\huge\textbf{Investigating the Effect of Temperature on the Rate Constant of the Decomposition of Aqueous Hydrogen Peroxide.}}
  \vspace{0.35in}

  \large January 29, 2026 \\
  \vspace{0.2in}

  \large IB  Chemistry HL \\
  \textit{Internal Assessment}

  \vspace{2.5in}
  \small \textit{I assert that this assessment is solely my authentic work. I understand that if this work does not meet the standards of my school's Academic Honesty Policy, it will not be submitted to the eCoursework system and I will receive no grade from the IBO.}

  \vspace{1in}
  \small Word count:\realwordcount / 3000
\end{titlepage}

\tableofcontents
\thispagestyle{empty}
\newpage

\doublespacing
\setcounter{page}{1}

\section{Introduction}
\label{sec:introduction}
\indent \indent The reaction of hydrogen peroxide, \ce{H2O2}, decomposes to form oxygen gas and water. Although this reaction is thermodynamically favourable to the products side, it will proceed at an extremely-slow rate at room temperature, unless catalyzed:

\begin{center}
  \ce{2H2O2_{(aq)} -> 2H2O_{(l)} + O2_{(g)}}
\end{center}

\noindent In this investigation, potassium iodide (\ce{KI}) was used as a catalyst so that the reaction (specifically, the formation of oxygen gas) inside the system canb occur at a measurable rate within a practical time frame. This investigation aims to explore how temperature impacts the rate constant for the catalyzed decomposition reaction of aqeuous hydrogen peroxide. The rate of the reaction is observed indirectly by measuring the initial rate of increase in gas pressure inside a sealed flask, as oxygen gas (\ce{O2_{(g)}}) builds up.

\subsection{Background Concepts}
\label{sec:introduction-background-concepts}

% \subsubsection{The Collision Theory.}
% \label{sec:introduction-background-concepts-collision-theory}
\indent \indent The collision theory states that for a chemical reaction can only occur when particles collide with energy and the correct orientation. As the temperature of the reaction system increase, reacting particles have greater average kinetic energy. This increase in the particle's collision ferquency and the increase in the probability of collisions with energy \textit{equal or greater than} the activation energy ($\text{E}_\text{A}$). Thus, increasing the temperature of the reaction is expected to increase the rate of the reaction and the reaction constant. The temperature dependence of a rate constant can be modelled by the Arrhenius equation:

\begin{equation}
  k = A \cdot e^{- \frac{E_{A}}{RT}}
\end{equation}

\noindent Where $k$ is the rate constant, $A$ is the frequency factor, $E_{A}$ is the activation energy, $R$ is the gas constant ($8.31\ \text{J}\ \text{K}^{-1}\ \text{mol}^{-1}$)\footnote{Value obtained from IB Chemistry Data Booklet, version 1.0 (first assessment 2025).}, and $T$ is the temperature, in Kelvins.

Taking the natural logarithm on both sides of the equation will yield a linear graph:

\begin{equation}
  ln(K) = - \frac{E_{A}}{RT} + ln(A)
\end{equation}

\noindent Where the slope of the linear graph is $- \frac{E_{A}}{R}$, and the y-intercept is $ln(A)$.

As oxygen gas (\ce{O2_{(g)}}) forms in the sealed flask of approximate constant gas volume, the pressure increases. Assuming ideal gas behaviour, the pressure should increase as $n$ (the number of moles of gas) increases, but $V$ (volume) stays constant:

\begin{equation}
  PV = nRT
\end{equation}

\begin{equation}
  \therefore P \propto n
\end{equation}

\noindent Given that the initial concentration of \ce{H2O2} and the amount of catalyst added were held constant across all trials, the calculated rate constant can be compared across temperature levels.

\begin{center}
  $\ast$~$\ast$~$\ast$
\end{center}

\noindent As a result, the research question for this investigation is:

\begin{center}
  \textit{How does the temperature affect the rate constant for the decomposition of aqueous hydrogen peroxide, determined from the initial rate of increase of gas pressure in a sealed flask?}
\end{center}

\subsection{Hypothesis}
\label{sec:introduction-hypothesis}
\indent \indent Given that the Arrhenius rate constant is dependent on the temperature, I hypothesize that increasing the temperature will cause the rate constant for the decomposition of aqueous hydrogen peroxide, \ce{H2O2}, to increase exponentially.

\section{Research Design}
\label{sec:research-design}

\subsection{Variables}
\label{sec:research-design-variables}

\noindent \underline{\textbf{Independent variable:}} Temperature of the reaction system. The \ce{H2O2} in the sealed flask was equilibrated to the target temperature using a water bath and heat plate before the start of each trial. The temperature was monitored using a Vernier Temperature Probe before and during the trials. The trials were only started once the water bath temperature reached within $1\ ^\circ\text{C}$ of the target temperature.

\

\noindent \underline{\textbf{Dependent variable:}} Rate of the decomposition reaction of \ce{H2O2} and \ce{KI}.

\

\noindent \underline{\textbf{Controlled variable:}} Initial concentration of \ce{H2O2} ($\text{mol}\ \text{dm}^{-3}$), total volume of gas space in the sealed flask, environmental conditions (temperature, humidity), time interval between pressure readings, the same apparatus setup (same Vernier pressure sensor used for all measurements), and data-collection settings.

\subsection{Materials}
\label{sec:research-design-materials}
\begin{itemize}
  \item (1x) Bottle of 3\% hydrogen peroxide, \ce{H2O2} (aqueous)
  \item (1x) Bottle of potassium iodide, \ce{KI} (solid)
  \item (1x) Jug of distilled water
  \item (1x) $1000\ \text{mL}$ beaker (waste beaker)
  \item (2x) $600\ \text{mL}$ beakers
  \begin{enumerate}
	\item (1x) for ice bath
	\item (1x) for \ce{H2O2} solution
  \end{enumerate}
  \item (1x) $250\ \text{mL}$ flask (for experiment, to be placed in water bath) 
  \item (1x) $250\ \text{mL}$ beaker (for \ce{H2O2} solution)
  \item (1x) $100\ \text{mL}$ beaker (for \ce{KI} solution)
  \item (1x) Glass bowl (for water bath)
  \item (1x) Rubber stoppers (with holes and pipes)
  \item (1x) Vernier Gas Pressure Sensor
  \item (1x) Vernier Stainless Steel Temperature Probe
  \item (1x) Vernier LabQuest
  \item (1x) $25\ \text{mL}$ Pipette + suction bulb
  \item (1x) $10\ \text{mL}$ graduated cylinder (for measuring distilled water)
  \item (1x) Digital balance
  \item (1x) Metal stick from ring stand
  \item (1x) Safety goggles
  \item (1x) Lab coat
\end{itemize}

\subsection{Experimental Design}
\label{sec:research-design-experimental-design}
\indent \indent This experiment was designed to isolate the temperature of the water bath as the sole independent variable affecting the rate constant for the \ce{KI}-catalyzed decomposition reaction of aqueous hydrogen peroxide. Oxygen gas produced during the reaction increased the pressure inside a sealed $250\ \text{mL}$ flask. Thus, the increase in the rate of the initial pressure value was used to determine the rate constant at each temperature. A water bath (wide glass bowl on heat plate) was used to equilibrate the reaction system to within $1 ^\circ \text{C}$ of the target temperature, with ice available in a beaker nearby to lower temperatures when required. The mass of \ce{KI} were pre-measured and recorded before data collection, and sealed with plastic wrap and rubber band.

\subsection{Procedure}
\label{sec:research-design-procedure}

\subsubsection{Part I: Setup}
\label{sec:research-design-procedure-setup}
\begin{enumerate}
  \item Obtain safety goggles and lab coat.
  \item Gather the required items from \textit{section \ref{sec:research-design-materials}} to work station.
  \item Assemble the pressure measurement setup:
  \begin{itemize}
    \item Connect the Vernier Gas Pressure sensor to the LabQuest
    \item Connect the airtight plastic tube to the pressure sensor.
    \item Connect the plastic tubing to the rubber stopper.
    \item Close the valve on the rubber stopper.
  \end{itemize}
  \item Secure the $250\ \text{mL}$ flask on the ring stand using a ring clamp, so that it remains stable during the data collection.
  \item Add $25\ \text{mL}$ of distilled water to the $250\ \text{mL}$ flask for setup purposes.
  \item To prepare the water bath, place the glass bowl on the hot plate and add water to a level so that it just submerges the $25\ \text{mL}$ water inside the flask. This will represent the hydrogen peroxide during the experiment.
  \item Draw a horizontal line on the neck of the reaction flask using a marker to standardize the insertion depth of the rubber stopper in every trial.
  \begin{itemize}
    \item \underline{\textbf{Note:}} Before inserting the rubber stopper each time, add a few drops of water on fingers, and rub the sides of the rubber stopper to improve the sealing.
  \end{itemize}
  \item Ensure the temperature probe is not touching the bottom of the glass bowl or the sides of the flask. Secure the temperature probe using a ring clamp (to standardize the placement between trials)
  \item Set the LabQuest to record at \textit{5 samples / second}, for a total of \textit{60 seconds} for each trial.
\end{enumerate}

\subsubsection{Part II: Preparation of materials}
\label{sec:research-design-procedure-preparation-of-materials}
\begin{enumerate}
  \item Pour a large amount of aqueous \ce{H2O2} into a clean $600\ \text{mL}$ beaker to use as the master solution for all trials. Excess can be poured back using a funnel.
  \item Prepare potassium iodide portions for each trial:
  \begin{enumerate}
    \item Using a digital scale, weigh $0.30\ \text{g}$ of solid \ce{KI} into individual plastic cups.
    \item Label each cup with its corresponding temperature level and trial number.
    \item Seal each cup using a piece of plastic film, and secure it with rubber band to reduce exposure to moisture from air.
  \end{enumerate}
  \item Using another clean $600\ \text{mL}$ beaker, fill with ice. This will be used to lower the temperature of the water bath for lower temperature levels.
\end{enumerate}

\subsubsection{Part III: Calibration}
\label{sec:research-design-procedure-Calibration}
\begin{enumerate}
  \item With the apparatus assembled, open the valve on the rubber stopper so that the pressure sensor is open to the atmosphere. Record the atmospheric pressure for $10\ \text{seconds}$ and record the average pressure in $\text{kPa}$.
  \item Rinse the $25\ \text{mL}$ pipette with aqueous \ce{H2O2} to avoid contaminatino, then dispose into the waste beaker.
\end{enumerate}

\noindent \underline{\textbf{Note:}}  An airtight-sealing calibration was not conducted because the experiment relies on finding the change in pressure over time. In the case of a minor leak, the pressure readings would be consistently lower than the actual pressure, which does not affect the rate of change in pressure and the rate constant.

\begin{figure}[H]
	\centering
	\includegraphics[width=14cm]{Resources/apparatus_setup.jpeg}
	\caption{Apparatus setup.}
	\label{fig:research-design-procedure-apparatus_setup}
\end{figure}

\subsubsection{Part IV: Experiment}
\label{sec:research-design-procedure-experiment}
\begin{enumerate}
  \item Adjust the water bath temperature to the target temperature, either by using the hot plate or by adding ice (depending on temperature level).
  \item Pipette $25\ \text{mL}$ of \ce{H2O2} to a clean $250\ \text{mL}$ beaker. Once the temperature of the water bath reaches within $1\ ^\circ\text{C}$ of the target temperature, pour the \ce{H2O2} into the reaction flask. Keep the reaction flask in the water bath to maintain consistency.
  \item Measure $5\ \text{mL}$ of distilled water using a $10\ \text{mL}$ graduated cylinder, and pour into the plastic cup containing the pre-weighed \ce{KI} for the trial. Stir to dissolve the \ce{KI}. Repeat twice, so that there is a total of $10\ \text{mL}$ of \ce{KI} solution.
  \item Start the data collection and collect a $3\ \text{second}$ baseline with the flask sealed. Then, remove the rubber stopper, and quickly add the \ce{KI} solution into the reactino flask, and immediately seal the flask with the rubber stopper (with water on sides of flask to improve seal).
  \item Swirl the flask gently for \textit{5\ \text{seconds}}, then keep the flask still for the remainder of the run.
  \item After data collection ends, save the run, and record the LabQuest run number in the data table with the corresponding trial.
  \item Dispose the solution into the waste beaker, and rinse the reaction flask once using tap water, then rinse again with distilled water.
  \item Repeat \textit{steps 1-7} for each trial at each temperature level, for a total of \textit{40 trials}.
\end{enumerate}

\subsubsection{Part V: Cleanup}
\label{sec:research-design-procedure-cleanup}
\begin{enumerate}
  \item Disconnect the gas pressure sensor and temperature probe from the LabQuest.
  \item Save the file on the LabQuest, and download the data to a computer.
  \item Neutralize the waste solution by adding a small amount of sodium thiosulfate, then dispose the waste solution down the drain with running water.
  \item Return used glassware to the dirty bin, and return all equipment.
\end{enumerate}

\noindent \underline{\textbf{Note:}} If table turns yellow, wipe down with a paper towel, some water, and a few crystals of sodium thiosulfate.

\section{Data Collection}
\label{sec:data-collection}

\subsection{Data Tables}
\label{sec:data-collection-data-tables}

\subsection{Sources of Uncertainty}
\label{sec:data-collection-sources-of-uncertainty}
\indent \indent To reduce random errors, 5 trials were conducted at each temperature level and averaged. Other variables, such as initial \ce{H2O2} concentration and volume, mass of \ce{KI}, apparatus setup, and sealing method were all kept constant across all runs so that differences in the initial pressure-increase rate can be attributed primary to temperature.

\section{Data Processing}
\label{sec:data-processing}

\subsection{Uncertainty Propagation}
\label{sec:data-processing-uncertainty-propagation}
\indent \indent The uncertainty values for temperature and pressure were obtained from specifications of the equipment used, and the smallest scale on the digital balance was used as the uncertainty of mass: $\pm 0.01\ \text{g}$. A $25\ \text{mL}$ volunmetric pipette was used to obtain the \ce{H2O2} for all trials, but the exact volumes delivered were not confirmed with a graduated cylinder. Thus, the pipette's tolerance was used as the volume uncertainty: $\pm 0.06\ \text{mL}$. Time was recorded automatically by the LabQuest, so its uncertainty was assumed negligible relative to pressure variation throughout the experiment.

\begin{table}[H]
	\centering
	\resizebox{15cm}{!}{
	\begin{tabular}{|>{\centering\arraybackslash}m{8cm}|
	                  >{\centering\arraybackslash}m{6cm}|} \hline
			
			\textbf{Tool} & \textbf{Uncertainty} \\ \hline
      Vernier Stainless Steel Temperature Probe & $\pm 0.1\ ^\circ \text{C}$ at $0\ ^\circ \text{C}$, $\pm 0.5\ ^\circ \text{C}$ at $100\ ^\circ \text{C}$ \\ \hline
      Vernier Gas Pressure Sensor & $\pm 4\ \text{kPa}$ \\ \hline
      Digital Balance & $\pm 0.01\ \text{g}$ \\ \hline
      Volunmetric Pipette & $\pm 0.06\ \text{mL}$ \\ \hline
		\end{tabular}
	}
	\caption{Summary of uncertainties by tool.}
	\label{tab:data-processing-uncertainty-propagation-summary-of-uncertainties-by-tool}
\end{table}

Given that the Vernier documentation provides two uncertainty values for the temperature probe, linear interpolation was used for the uncertainty of the device:

\begin{equation}
  \Delta{T}(x) = 0.20 + \frac{0.50 - 0.20}{100} \cdot x + 273.15
\end{equation}

\noindent Where $x$ is the measured temperature obtained from the temperature probe (in Celsius), and $\Delta{T}$ is the uncertainty in Kelvin.

\subsubsection{Uncertainty in rate constant}
\label{sec:data-processing-uncertainty-propagation-uncertainty-in-rate-constant}


\subsection{Processed Data}
\label{sec:data-processing-processed-data}

\noindent \underline{\textbf{Note:}} The initial rate of pressure increase was calculated by finding the slope of the trendline of the pressure-time graph, using a fixed window of \textbf{\textit{15 seconds}} starting from the moment the pressure sensor settles and reaction begins.
\subsection{Raw Data}
\label{sec:data-processing-raw-data}

\subsection{Example Calculation}
\label{sec:data-processing-example-calculation}

\section{Data Analysis}
\label{sec:data-analysis}

\section{Conclusion}
\label{sec:conclusion}

\section{Evaluation}
\label{sec:evaluation}

\subsection{Strengths}
\label{sec:evaluation-strengths}

\subsection{Weaknesses \& Limitations}
\label{sec:evaluation-weaknesses-and-limitations}

%Bibliography
%TC:ignore
\newpage
\nocite{*}
\bibliographystyle{unsrtnat}
\bibliography{ref}

%Appendix
\newpage
\appendix
\section{Appendix}
\label{sec:appendix}

\subsection{Source Code}
\label{sec:appendix-ia-cam-recorder-fan}
\indent \indent All source code and raw data are open-source, available to view on GitHub.

%End
%TC:endignore
\thispagestyle{empty}
\end{document}
